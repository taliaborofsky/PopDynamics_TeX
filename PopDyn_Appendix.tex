\documentclass[11pt, oneside]{article}   	% use "amsart" instead of "article" for AMSLaTeX format
\usepackage{PopDyn}
\externaldocument{PopDynSocialLearning}

\begin{document}
\appendix
\renewcommand{\thesection}{A\arabic{section}}
\numberwithin{equation}{section}
\setcounter{equation}{0}
\numberwithin{figure}{section}
\setcounter{figure}{0}
\numberwithin{table}{section}
\setcounter{table}{0}

 
 
 
 
 \section{Calculation of mean population fitness} \label{Derive_W}
 $W$, the mean population fitness, is the sum of the right sides of Eqns. \eqref{recursions}, i.e.,
\begin{align*}
W &= 1 + r(\tilde{u}_r + \tilde{x}_r)+ R(\tilde{u}_R + \tilde{x}_R), \\
\intertext{and from \eqref{L_u}, \eqref{L_x}, and \eqref{rec_juveniles},}
W&= 1 + r \lrb{u L_u(p_r,r) + x \lrp{ L_u(p_r,r) + \dk p_r + \dpc \frac{r}{r+R} }} \\
& \hspace{1cm}+ R\lrb{u(1 - L_u(p_r,r)) + x \lrp{ 1 - L_u(p_r,r) - \dk p_r - \dpc \frac{r}{r+R} }}.
\end{align*}
Since $u + x = 1$,
\begin{align} \label{eq_W_all_app}
W&= 1 + r \lrb{L_u(p_r,r) + x \lrp{ \dk p_r + \dpc \frac{r}{r+R}} } \notag \\
&\hspace{1cm} + R \lrb{ 1 - L_u(p_r,r) -x \lrp{ \dk p_r + \dpc \frac{r}{r+R} }}, \\
 &= 1 + R + (r - R) \lrb{ L_u(p_r,r)+ x \lrp{ \dk p_r + \dpc \frac{r}{r+R}} } \notag.
\end{align}

\section{Proof of Result \ref{Order_R_delta_r}}. \label{Proof_Order_R_delta_r}

We first prove part (i). From (\ref{sys_allB}a), $W - \delta = 1$ because $\hN_p > 0$. Then from Eqn. \eqref{before_delta_eq}, $\delta = R$ is only possible if $L(\hu_r,\hr) = 0$ or $\hr = R$. If $L(\hu_r,\hr) = 0$, assuming $\pc > 0$, then from \eqref{L_u}, $\hr = K\hu_r = 0$.

Parts (ii) and (iii) follow from Eq. \eqref{delta_eq} and the requirement that $0 \leq L(\hu_r,\hr) \leq 1$ because $L(\hu_r,\hr)$ is a probability.
\subsection{Note on the Rare Case of $\hr = R = \delta$}

Note that in the rare case of $\hr = R = \delta$, from \eqref{sys_allB}b
$$
(1 + \delta)\hu_r = L(\hu_r, \hr)(1+\delta)
$$
 so 
 $$
\hu_r = L(\hu_r,\hr) = K \hu_r + \pc/2
$$
and thus
\begin{equation} \label{result3_4}
\hu_r = \frac{1}{2} \lrp{ \frac{\pc}{1-K} }.
\end{equation}

If there is no time delay, i.e. the predation term is $P(r,u_r) = L_{total}$ as defined in \eqref{L_total}, and $\hr = \delta$, then the equilibrium population size is
$$
\hN_p = \frac{1-\delta}{\beta L(\hu_r, \hr)}
$$
which simplifies to
$$
\hN_p = \frac{2(1-\delta)(1-K)}{\beta \pc}.
$$
If there is a time delay, i.e. the predation term is $p_r = u_r$, and $\hr = \delta$, then because $\hr = 1 - \beta \hN_p \hu_r$, the equilibrium forager population size is
$$
\hN_p = \frac{1 - \delta}{\beta \hu_r},
$$
which, from \eqref{result3_4}, simplifies to
\begin{equation}\label{hNp_delay}
\hN_p = \frac{2(1-\delta)(1-K)}{\beta \pc}.
\end{equation}

\section{Proof of Result \ref{Result_eq_Rlessthandelta} } \label{Proof_eq_Rlessthandelta}

If $\hr > 0$, then since $0 < L_u(\hu_r,\hr) \leq 1$ and $R < \delta$, from \eqref{delta_eq} we must have $\hr \geq \delta$.
Substituting \eqref{delta_eq} for $L_u(\hu_r, \hr)$ and $W = 1 + \delta$ into (\ref{sys_allB}b),
\begin{equation}\label{hu_r}
\hu_r = L_u(\hu_r, \hr) \frac{1 + \hr}{1 + \delta} = \frac{(\delta - R)(1+\hr)}{(\hr-R)(1+\delta)}.
\end{equation}
Since $\hr \geq \delta > R$, \eqref{hu_r} is legitimate, i.e. $0 < \hu_r \leq 1$, if
%$$
%\frac{(\delta - R)}{(\hr-R)} \leq \frac{1 + \delta}{1+\hr}
%$$
%which can be rearranged and simplified to
$$
(\delta - \hr)(1 + R) \leq 0
$$
which is true because $\hr \geq \delta$.

Thus at the equilibrium, from Eqns. \eqref{L_u} and \eqref{delta_eq},
\begin{align} \label{beforeQ_r}
 K \frac{(\delta - R)(1+r)}{(r-R)(1+\delta)} + \pc \frac{r}{r+R} = \frac{\delta - R}{r-R}.
\end{align}
Equation \eqref{beforeQ_r} can be rewritten as $Q_r(r) = 0$, where 
\small
\begin{equation} \label{app_Q_r}
Q_r(r) = r^2 \lrb{\pc + \frac{K(\delta-R)}{(1+\delta)} } +r \lrb{ (\delta-R) \lrp{ \frac{K(1+R)}{1+\delta}-1}- R \pc} -R(\delta-R)\lrp{1 - \frac{K}{1+\delta}},
\end{equation}
\normalsize
which we write as $Q_r(r) = Ar^2 + Br + C$.
Since 
$$
Q_r(0) = C = -R(\delta-R) \lrp{1 - \frac{K}{1+\delta}} < 0,
$$
there is an equilibrium $\hr \geq \delta$ if $Q_r(1) \geq 0$ and $Q_r(\delta) \leq 0$, where
\begin{equation} \label{Q_r_1}
Q_r(1) = A + B + C = \pc (1 - R) + (\delta - R)(1 +R) \lrp{\frac{2K}{1+\delta}-1} 
\end{equation}

and \red{(I showed all my steps so it's easier to check)}
\begin{align} \label{Q_r_delta}
Q_r(\delta) &= A \delta^2 + B \delta + C \\
&= (\delta-R) \lrb{\frac{K \delta^2}{1+\delta} + \delta \lrp{ \frac{K(1+R)}{1+\delta} - 1 } - R \lrp{1 - \frac{K}{1+\delta}} } + \delta^2 \pc - R \pc \delta \notag \\
&= (\delta-R) \lrb{ \frac{K\delta^2}{1+\delta} - (R + \delta) \lrp{1 - \frac{K}{1+\delta}} + \frac{KR \delta}{1+\delta} + \pc \delta } \notag \\
&= (\delta-R) \lrb{ (\delta + R) \lrp{ \frac{K\delta}{1+\delta} - 1 + \frac{K}{1+\delta} } + \pc \delta } \notag \\
&= (\delta - R) \lrb{ (\delta + R) \lrp{ \frac{K\delta + K}{1+\delta} - 1} + \pc \delta } \notag \\
&= (\delta-R) \lrb{(\delta+R)(K-1) + \pc \delta} \notag \\
&= (\delta - R) \lrb{ \delta(K + \pc - 1) + R(K-1)} \notag \\
&= (\delta - R) \lrb{- \pw \delta + R(K - 1)}.
\end{align}
However, $Q_r(\delta) \leq 0$ because $(\delta - R) > 0$ and $- \pw \delta + R(K-1) \leq 0$. Thus there is an equilibrium $\delta \leq \hr \leq 1$ if $Q_r(1) \geq 0$.

To complete the proof, we must show there is a legitimate equilibrium predator population size $\hN_p$. No time delay entails $\hr = 1 - \beta \hN_p L_u(\hu_r, \hr)$ and a time delay entails $\hr = 1 - \beta \hN_p \hu_r$.  If there is no time delay and $\beta > 0$,
\begin{equation} \label{app_hNp_nodelay}
\hN_p = \frac{(1 - \hr)(\hr-R)}{\beta(\delta-R)},
\end{equation}
where $L(\hu_r, \hr)$ is given by \eqref{delta_eq} and $\hr$ is the larger root of $Q_r(r)$.

If there is a time delay and $\beta > 0$,
\begin{equation}
\hN_p = \frac{1 - \hr}{\beta \hu_r},
\end{equation}
or from \eqref{hu_r}
\begin{equation}
\hN_p = \frac{(1-\hr^2)(\delta-R)}{\beta (\hr-R)(1+\delta)}.
\end{equation}

\section{Proof of Result \ref{Result_Rgreaterthandelta}}\label{Proof_Rgreaterthandelta}

If $R > \delta$, then we know $\hr \leq \delta < R$, so $\hr > 0$ exists if $Q_r(r)$ has at least one root between $r = 0$ and $r = \delta$. 
Note that
$$
Q_r(0) = C = -R(\delta - R) \lrp{1 - \frac{K}{1+\delta}} \geq 0,
$$
and from \eqref{Q_r_delta}, $Q_r(\delta) \geq 0$ because $R> 0$. Thus if $A < 0$, i.e. the parabola of $Q_r(r)$ opens down, there is no equilibrium with $\hr > 0$. 

On the other hand, if the parabola $Q_r(r)$ points upward, i.e. $A > 0$, then $Q_r(r)$ can only have roots within the range $0 < \hr \leq \delta$ if $Q_r'(0) = B < 0$ and $Q_r'(\delta)  = 2A \delta + B> 0$.
The roots of $Q_r(r)$ are
$$
\hr = \frac{-B \pm \sqrt{ B^2 - 4 AC}}{2A}.
$$
where, if the discriminant $B^2 -4AC > 0$, then there are two $\hr > 0$ equilibria. If the discriminant is instead nonnegative, then there are no real roots of $Q_r(r)$. 

\section{Proof of Result \ref{Result_EqPreyExtinction}} \label{Proof_EqPreyExtinction}

Given $\pc > 0$, if $\hr = 0$ then at equilibrium $W u_r= K u_r$, so either $W = K$ or $\hu_r = 0$. Say $W = K$ and $\hu_r \neq 0$. Then from \eqref{W_allB},
$$
K = 1 + R(1-K u_r)
$$
which is a contradiction because $0 \leq K < 1$, leaving us with $\hu_r = \hr = 0$. 

Next, we aim to show that if $\hu_r = \hr = 0$, then $\hN_p = 0$ if $R < \delta$. The mean population fitness is $W = 1 + R$ because $\hu_r = \hr = 0$. Since $R < \delta$, $W \neq 1 + \delta$, and thus the only possible equilibrium is $\hN_p = 0$.

For the last part of the proof, if $R > \delta$, then $N_p$ would increase infinitely because $W > 1 + \delta$, so $N_p' > N_p$.

\section{Proof of Result \ref{Result_EqPredExtinct}} \label{Proof_EqPredExtinct}
From (\ref{sys_allB}b), at this equilibrium $\hw \hu_r = 2 L(\hu_r,\hr)$ where $L(\hu_r,\hr) = K \hu_r + \frac{\pc}{1+R}$ and, from \eqref{W_allB}, $\hw = 1+R+(1-R)L(\hu_r,\hr)$. Then
\begin{align} 
u_r \lrb{1 + R + (1-R)\lrp{Ku_r + \frac{\pc}{1+R}}} = 2 \lrp{K u_r + \frac{\pc}{1+R}} \notag \\
\intertext{or}
Q_u(u_r) = u_r^2 K(1-R) + u_r\lrb{ 1+R + (1-R) \frac{\pc}{1+R} -2K} - 2 \frac{\pc}{1+R} = 0. \label{E1_u}
\end{align}
Thus $\hu_r$ is the larger root of $Q_u(u_r)$ because $K(1-R)>0$, $Q_u(0) < 0$, and
$$
Q_u(1) = -(1-R)\lrp{K+\frac{\pc}{1+R}} (1+R) > 0.
$$

\section{Derivation of Internal Stability Jacobian} \label {app_Jstar_derive}

Near the equilibrium $(\hN_p, \hu_r, \hr)$,
\begin{align*}
L_u(\hu_r + \du,\hr+\dr) &= K( \hu_r + \du) + \frac{\hr+\dr}{\hr+\dr+R} \pc \\
&= L_u(\hu_r, \hr) + \D_L
\end{align*}
where
\begin{equation}\label{D_L}
\D_L \approx K \du + \frac{R \pc \dr}{(\hr + R)^2}
\end{equation}
because
$$
\frac{\hr + \dr}{\hr + \dr + R } \approx (\hr + \dr) \lrp{1 - \frac{\dr}{\hr + R}} \lrp{\frac{1}{\hr + R}} \approx \frac{\hr}{\hr + R} + \frac{\dr}{\hr + R} \lrp{1 - \frac{\hr}{\hr + R}}
$$
and $1 - \frac{\hr}{\hr + R} = \frac{R}{\hr + R}$. The mean population fitness near the equilibrium is
\begin{align}
W + \dw &= 1 + R + (\hr + \dr - R) L_u(\hu_r + \du,\hr+dr)\notag \\
\dw & \approx \D_L (\hr - R) + \dr L_u(\hu_r, \hr) \label{dw_dl}
\intertext{which simplifies to}
\dw &= K (\hr-R) \du + \dr \lrb{ L(\hu_r, \hr) + \frac{R \pc (\hr - R)}{(\hr + R)^2} } \label{dw_simplified}.
\end{align}

The predator population size near the  equilibrium $\hN_p$ is
$$
\hN_p + \dnp' = (\hN_p + \dnp)(\hw + \dw - \delta) 
$$
so
\begin{align} \label{D_N}
\dnp' &\approx \D_W \hN_p + \dnp (W - \delta)  \notag\\
&=  \dnp (\hw - \delta)+ \hN_p \lrc{ K(\hr - R) \du + \dr \lrb{ L(\hu_r, \hr) + \frac{R \pc (\hr - R)}{(\hr + R)^2} }}.
\end{align}


The frequency of predators exploiting the CP near the equilibrium is
\begin{align*}
\hu_r + \du' &= \frac{1}{\hw + \dw} (L(\hu_r, \hr) + \D_L)(1 + \hr + \dr) \\
\intertext{so}
\hu_r + \du' & \approx \frac{1}{\hw} \lrp{1 - \frac{\dw}{\hw}} \lrb{ L(\hu_r, \hr)(1+\hr)  + \dr L(\hu_r, \hr) + \D_L (1+\hr)}\\
&= \lrp{1 - \frac{\dw}{\hw}} \lrb{ \hu  + \dr \frac{L(\hu_r, \hr)}{\hw} + \D_L \frac{(1+\hr)}{\hw}},\\
\intertext{and thus the perturbation from equilibrium is}
\du' &\approx \dr \frac{L(\hu_r, \hr)}{\hw} + \D_L \frac{(1+\hr)}{\hw} - \frac{\hu_r}{\hw} \dw,
\end{align*}
which, after substituting \eqref{D_L} for $\D_L$ and $\hw = 1 + \delta$, becomes
\begin{align} \label{D_u}
\du'  &= \du \frac{K}{\hw} \lrp{1 + \hr - \hu_r(\hr-R)} \notag \\
&\ \ + \dr\lrc{\frac{L(\hu_r, \hr)}{\hw}(1-\hu_r) + \frac{\pc R}{\hw(\hr+R)^2} \lrb{1 + \hr - \hu_r(\hr-R)}}
\end{align}

\subsubsection{Internal Stability, no time delay}

The CP relative density near equilibrium is
\begin{align*}
\hr + \dr' &= \frac{(\hr + \dr)\lrb{2 - \beta (\hN_p+ \dnp)(L(\hu_r, \hr) + \D_L)}}{1 + \hr + \dr} \\
&\approx \frac{(\hr + \dr) \lrp{2 - \beta \hN_pL(\hu_r, \hr) -\beta \dnp L(\hu_r, \hr) - \beta \hN_p\D_L}}{1 + \hr + \dr}.
\end{align*}
To simplify, note that $\frac{1}{1 + \hr + \dr} \approx \frac{1}{1 + \hr} \lrp{1 - \frac{\dr}{1 + \hr}}$. Then
\begin{align*}
\hr + \dr' &= \frac{\hr (2 - \beta \hN_p L(\hu_r,\hr)-\hr \beta \lrp{ L(\hu_r,\hr) \dnp + \hN_p \D_L} + \dr \lrp{2 - \beta \hN_p L(\hu_r,\hr)}}{1+\hr} \lrp{1 - \frac{\dr}{1+\hr}}. 
\end{align*}
\red{Additional steps are shown in red:
\begin{align*}
\hr + \dr' &= \lrp{ \hr  - \frac{\hr \beta (\dnp L(\hu_r, \hr) + \hN_p\D_L)}{1+\hr} + \dr \frac{2 - \beta \hN_p L(\hu_r,\hr)}{1+\hr} } \lrp{1 - \frac{\dr}{1+\hr}} \\
\dr' &\approx -\dr \frac{\hr}{1+\hr} - \frac{\hr \beta (\dnp L(\hu_r, \hr) + \hN_p \D_L)}{1+\hr} + \dr \frac{2 - \beta \hN_p L(\hu_r,\hr)}{1+\hr}, \\
\end{align*}}
and substituting \eqref{D_L} for $\D_L$ gives
\small
\begin{equation} \label{D_r_nodelay}
\dr'  \approx - \dnp \frac{\hr \beta L(\hu_r, \hr)}{1+\hr} - \du \frac{K\hN_p\hr \beta}{1+\hr} + \lrp{\frac{2-\beta \hN_p L(\hu_r,\hr)-\hr}{1+\hr} - \frac{\hN_p\hr \beta R \pc}{(1+\hr)(\hr+R)^2} } \dr .
\end{equation}
\normalsize

Using Eqs \eqref{D_N}, \eqref{D_u}, and \eqref{D_r_nodelay}, the local stability matrix for the equilibrium $\hN_p, \hu_r, \hr$ is of the form
\begin{equation} \label{Jstar_nodelay}
J^* = \begin{pmatrix}
\hw-\delta & a & b \\
0 & c & d \\
e & f & g
\end{pmatrix},
\end{equation}
where $a$, $b$ are the coefficients of $\du$ and $\dr$, respectively, from \eqref{D_N}, $c$ and $d$ are the coefficients of $\du$ and $\dr$, respectively, from \eqref{D_u}, and $e, f,$ and $g$ are the coefficients of $\dnp, \du,$ and $\dr$, respectively, from \eqref{D_r_nodelay}.

\section{Proof of Result \ref{result_E0_nodelay} } \label{Proof_E0_nodelay}
At E0, i.e. $\hN_p = \hr = \hu_r = 0$, the Jacobian is
\begin{equation}
J^*_{E0} = \begin{pmatrix}
\hw - \delta & 0 & 0 \\
0 & \frac{K}{\hw} & \frac{\pc}{\hw R} \\
0 & 0 & 2
\end{pmatrix}
\end{equation}
where from \eqref{W_allB} $\hw = 1 + R$. The eigenvalues are thus $1 + R - \delta, \frac{K}{1+R},2$, so E0 is unstable.

However, consider the situation in which the predator population has depleted the CP, i.e. the system has $r = 0$. Then along this null-cline, if $R < \delta$, then E0 is stable because from \eqref{W_allB}, $W = 1 + R(1 - L(u_r,0)) < 1 + \delta$ and thus $N' < N$. 

Now say we start from a point off the nullcline $r = 0$, where $N_p$, $u_r$, and $r$ are very small. Then from \eqref{r_allB},
$$
\frac{r'}{r} = \frac{2 - \beta N_p P(r,u_r)}{1+r} \approx 2 
$$
so $r' > r$.

For the second part of the proof, we look at the Jacobian $J^*$ from \eqref{Jstar_nodelay} at some point alone the nullcline $r = 0$ for $R < \delta$. Here,
\begin{equation} \label{Jstar_nodelay_r0nullcline}
J^*(r=0) = \begin{pmatrix}
W - \delta & a & b \\
0 & c & d \\
0 & 0 & g
\end{pmatrix}
\end{equation}
because $e = f = 0$. The eigenvalues are $W - \delta$, $c$, and $g$, where $W - \delta < 1$ because $R < \delta$,
\begin{equation}
c = \frac{K(1 + u_r R}{W}= \frac{K(1+u_rR)}{1+R(1-Ku_r)} 
\end{equation}
is less than one if $R > - \frac{1-K}{1-2K u_r}$, and
\begin{equation}
g = 2 - \beta N_p K u_r
\end{equation}
is less than one if $1 - \beta N_p K u_r < 0$.

\section{Proof of Result \ref{E1_nodelay}} \label{Proof_E1_nodelay}
At E1, i.e. $\hN_p = 0$, $\hr = 1$, and $\hu_r$ is the larger root of \eqref{E1_u}, the Jacobian is
$$
J^*_{E1} = \begin{pmatrix}
\hw - \delta & 0 & 0 \\
0 & \frac{K}{\hw}(2 - \hu_r (1-R)) & \frac{L(\hu_r,1)}{\hw}(1-\hu_r) + \frac{\pc R (2 - \hu_r(1-R))}{\hw(1+R)^2} \\
- \frac{1}{2} \beta L(\hu_r,1) & 0 & \frac{1}{2} 
\end{pmatrix}
$$
with eigenvalues $\lambda_1 = \hw - \delta, \lambda_2 = \frac{K}{\hw}(2 - \hu_r(1-R)),$ and $\lambda_3 = 1/2$. From \eqref{W_allB},
$$
\hw = 1 + R + (1-R)L(\hu_r,1)
$$
so $\lambda_1 < 1$ if $R-\delta + (1-R) L(\hu_r,1) < 0$. Furthermore, $\lambda_2 < 1$ if
$$
K(2 - \hu_r(1-R)) < 1 + R + (1-R)L(\hu_r,1).
$$
which can be written as
$$
2K (1 - \hu_r (1-R)) < 1 + R + (1-R) \frac{\pc}{1+R}.
$$
\section{Derivation of Internal Stability Jacobian with Time Delay} \label{App_Jac_TimeDelay}

The equations for $\du'$ and $\dnp'$ are the same, but equation \eqref{D_r_nodelay} changes. Instead,
$$
\hr + \dr'  \approx \lrp{ \hr - \frac{\hr \beta (\dnp \hu_r + \hat{N}_p \du) }{1+\hr} + \dr \frac{2 - \beta \hN_p \hu_r}{1+\hr} } \lrp{1 - \frac{\dr}{1+\hr}}
$$
which simplifies to,
\begin{equation} \label{D_r_delay}
\dr' \approx \dnp \lrp{- \frac{\hr \beta \hu_r}{1+\hr}} + \du \lrp{ - \frac{\hr \beta \hN_p}{1+\hr}} + \dr \lrp{ \frac{2 - \beta \hN_p \hu_r - \hr}{1+\hr}}
\end{equation}

The Jacobian formed by Eqs \eqref{D_N}, \eqref{D_u}, and \eqref{D_r_delay}, is of the form
\begin{equation}
J^* = \begin{pmatrix} \label{Jstar_delay}
\hw-\delta & a & b \\
0 & c & d \\
e & f & g,
\end{pmatrix}
\end{equation}
where we redefine $a, b, c, d, e, f,$ and $g$ to be
\begin{align} \label{Jstar_constants_delay}
a &= \hN_p K, &b = \hN_p \lrb{L(\hu_r,\hr) + \frac{R \pc(\hr-R)}{(\hr+R)^2} }, \\
c&= \frac{K}{\hw} \lrp{1 + \hr - \hu_r(\hr-R)}, & \\
d &= \frac{L(\hu_r,\hr)}{\hw}(1-\hu_r) + \frac{\pc R \lrb{1+\hr - \hu_r(\hr-R)}}{\hw(\hr+R)^2}, &\\
e& = \frac{- \hr \beta \hu_r}{1+\hr} & f = \frac{- \beta \hr \hN_p}{1+\hr}, \\
 g &=  \lrp{ \frac{2-\beta \hN_p \hu_r - \hr}{1+\hr}}.
\end{align}
For equilibrium E2, if $\hN_p > 0$, then $\hN_p$ is defined in \eqref{hNp_delay}, $\hu_r$ is \eqref{hu_r}, $\hr$ is the solution to \eqref{Q_r}, and $\hw = 1 + \delta$. The eigenvalues are defined using equations \eqref{evals_constants} and \eqref{Jstar_evals}, where we substitute the new definitions of $a,b,c,d,e,f$, and $g$ from \eqref{Jstar_constants_delay} into \eqref{evals_constants}. 

\section{External Stability Jacobian} \label{App_External_Stability}

Near the equilibrium $\hN_p,\hu_r,\hr$,
\begin{subequations}
\begin{align}
\dx{r}' &\approx \frac{1}{\hw} \dx{r} L_x(\hu_r, \hr)(1+\hr) \\
\dx{R}' &\approx \frac{1}{\hw} \dx{R} (1 - L_x(\hu_r,\hr))(1+R)
\end{align}
\end{subequations}
so that the Jacobian determining local stability is of the form
\begin{equation}
J_x = \begin{pmatrix}
a_x & a_x \\
b_x & b_x
\end{pmatrix}
\end{equation}
for $a_x =\frac{1}{\hw} L_x(\hu_r,\hr)(1+\hr)$ and $b_x =\frac{1}{\hw} (1 - L_x(\hu_r,\hr))(1+R)$. There is one nonzero eigenvalue for this matrix, and it is
\begin{equation}
\lambda_x = a_x + b_x = \frac{1}{\hw}\lrb{1+R+ L_x(\hu_r,\hr)(\hr-R)}
\end{equation}
which simplifies to
\begin{equation}
\lambda_x = 1 + \frac{1}{\hw} \lrp{ \dk \hu_r + \dpc \frac{\hr}{\hr + R}}(\hr-R).
\end{equation}


\end{document}