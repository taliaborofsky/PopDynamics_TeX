\documentclass[11pt, oneside]{article}   	% use "amsart" instead of "article" for AMSLaTeX format
\usepackage{PopDyn}
\externaldocument{PopDynSocialLearning.tex}
\setcounter{section}{3}
\begin{document}

Page of results
\begin{lemma} \label{Result_L_delta}
If $\hN_p > 0$ at equilibrium and $R \neq \hr$, then
\begin{equation} \label{delta_eq}
L_u(\hu_r,\hr) = \frac{\delta-R}{\hr-R}.
\end{equation}
\end{lemma}


\begin{lemma}
If $R < \delta$, then there is one nonzero $\hu_r, \hr, \hN_p$ equilibrium if $\beta > 0$ and
\begin{equation} \label{exists_eq_Rlessthan}
\pc(1-R) + (\delta-R)(1+R) \lrp{ \frac{2K}{1+\delta} -1 } >0.
\end{equation}
\end{lemma}

\begin{lemma}
If $R > \delta$, then there exist two equilibria with $\hr > 0$ provided $Q_r''(r) > 0$, $-\delta Q_r''(r) < Q_r'(0)< 0$, and
\begin{equation}
(Q_r'(0))^2 - 2 Q_r''(r) Q_r(0) > 0.
\end{equation}
 Otherwise, no $\hr >0$ equilibria exist. 
\end{lemma}


\begin{lemma}
If $\hr = 0$ then $\hu_r = 0$, assuming $\pc > 0$, and if $R < \delta$, then $\hN_p = 0$.
\end{lemma}

\begin{lemma}
If $\hN_p= 0$ and $\hr = 1$, then there is one possible equilibrium frequency $\hu_r>0$ of predators hunting the CP.
\end{lemma}

%% E0, no delay
\begin{lemma} \label{result_E0_nodelay}
E0 is only stable along the nullcline $r = 0$ if $R < \delta$. Any point near the nullcline $r = 0$ will go to the nullcline if $R < \delta$, $R > - \frac{1-K}{1-2K u_r}$, and $1 - \beta N_p K u_r < 0$.
\end{lemma}

%%% E1, no delay
\begin{lemma} \label{E1_nodelay}
Without a time delay, E1 is stable if 
\begin{equation}
R-\delta + (1-R) L(\hu_r,1) < 0
\end{equation}
 and 
\begin{equation}
2K(1 - \hu_r(1-R)) < 1 + R + (1-R)\frac{\pc}{1+R}
\end{equation}
where $\hu_r$ is the larger root of $\eqref{E1_u}$.
\end{lemma}


Time Delay Stability Results:
%%% Delay, E0
\begin{lemma}
Result \ref{result_E0_nodelay} holds when there is a delay, i.e. E0 is unstable except along the $r = 0$ nullcline.
\end{lemma}

\begin{lemma}
When there is a time delay, E1 is unstable under the same conditions as in the situation with no time delay (Result \ref{E1_nodelay}).
\end{lemma}

Irrespective of time delay:
\begin{lemma}
The resource depletion constant $\beta$ does not affect the social-learning cut-off $s^*$ that maximizes $\hN_p$ for any combination of the parameters $\mu, \delta, R$.
\end{lemma}

\begin{lemma}
If $s = 0$, i.e. all predators are individual learners, then social learning evolves if $R < \delta$.
\end{lemma}

\end{document}